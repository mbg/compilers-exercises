\documentclass[10pt,a4paper]{exam} % turn it into a class! 
\usepackage[latin1]{inputenc}
\usepackage{amsmath}
\usepackage{amsfonts}
\usepackage{amssymb}
\usepackage{graphicx}
\usepackage{titlesec}
\usepackage{hyperref}
\usepackage{fancyeq}
\usepackage{tikz}
%\usepackage{tikz-uml}
\usepackage{mathpartir}
\usetikzlibrary{matrix,decorations.pathmorphing,shapes,arrows,backgrounds,positioning}
\usepackage{graphicx,xcolor}
\usepackage{geometry}
\usepackage{everysel}
\usepackage[normalem]{ulem}
\usepackage{mdframed}

\tikzset{
  treenode/.style = {align=center, inner sep=0pt, text centered,
    font=\sffamily},
  arn_n/.style = {treenode, circle, black, font=\sffamily\bfseries, draw=black,
    fill=white, text width=1.5em},% arbre rouge noir, noeud noir
  arn_r/.style = {treenode, circle, red, draw=red, 
    text width=1.5em, very thick},% arbre rouge noir, noeud rouge
  arn_x/.style = {treenode, rectangle, draw=black,
    minimum width=0.5em, minimum height=0.5em}% arbre rouge noir, nil
}

\usepackage[sc]{mathpazo}
\linespread{1.05}         % Palatino needs more leading (space between lines)
\usepackage[T1]{fontenc}

% some format settings
% for hard-bound final submission, use:
%\setlength{\oddsidemargin}{4.6mm}     % 30 mm left margin - 1 in
% for soft-bound version and techreport, use instead:

\setlength{\oddsidemargin}{-0.4mm}    % 25 mm left margin - 1 in
\setlength{\evensidemargin}{\oddsidemargin}
\setlength{\topmargin}{-5.4mm}        % 20 mm top margin - 1 in
\setlength{\textwidth}{160mm}         % 20/25 mm right margin
\setlength{\textheight}{237mm}        % 20 mm bottom margin
\setlength{\headheight}{5mm}
\setlength{\headsep}{5mm}
\setlength{\parindent}{0mm}
\setlength{\parskip}{\medskipamount}
\renewcommand\baselinestretch{1.2} % thesis format (not needed for techreport)
% don't let large figures hijack entire pages
\renewcommand\topfraction{.9}
\renewcommand\textfraction{.1}
\renewcommand\floatpagefraction{.8}

\pagestyle{headandfoot}
%\pointsinrightmargin
%\pointname{ marks}
%\marginpointname{ marks}

\marksnotpoints 

\definecolor{campurple}{HTML}{862D91} 
\definecolor{campurpledark}{HTML}{2A185C}

\hypersetup{  
  urlcolor=campurple,
  linkcolor=campurple,
  colorlinks=true  
}

\titlelabel{\llap{\thetitle\quad}}

\newcommand {\lbrac} {\makebox[0pt]{[\kern-1ex[}}
\newcommand {\rbrac} {\makebox[0pt]{]\kern-1ex]}}
\newcommand{\denote}[1]{\lbrac~#1~\rbrac}


\def\mystrut(#1,#2){\vrule height #1pt depth #2pt width 0pt} 

\titlespacing*{\section}{0pt}{0pt}{0pt}


\begin{document}

\newcommand{\course}{Compiler Construction}
\newcommand{\week}{I}
%\newcommand{\topics}{Data, Subtyping \& Objects, Semantic Equivalence, and Concurrency}

\everymath{\color{campurpledark}}
\everydisplay{\color{campurpledark}}

%\vspace{15pt}

%\begin{center}
%\emph{Complete SECTION 1 and ONE other section.}
%\end{center}

%\begin{center}
%\emph{Answer SECTION 1 and TWO other sections.}
%\end{center}

\marksnotpoints
\pointsdroppedatright
\marksnotpoints
\marginpointname{ \points}

\begin{center}
\LARGE {\textbf{\color{campurpledark} \course} }\\[-0.2cm]
\Large \color{campurpledark} Language Specification\\
\end{center}

{\color{campurple}\hrule}

\newcommand{\metavar}[1]{{\color{campurple}#1}}

\vspace{0.5cm}

\newcommand{\terminal}[1]{\texttt{\color{campurple}#1}}
\newcommand{\bl}[1]{{\color{black}#1}}


\begin{figure}[h]
\begin{mdframed}
    \begin{displaymath}
    \begin{array}{llcll}
    \multicolumn{5}{c}{\begin{array}{cc}
    \mathit{var} \in \texttt{[a-z][a-zA-Z0-9]*} & \mathit{ctr} \in \texttt{[A-Z][a-zA-Z0-9]*}
    \end{array}}\\\\
    \text{\bl{Program}} & \mathit{prog} & \bl{\to} & \mathit{binds} \\
    \text{\bl{Bindings}} & \mathit{binds} & \bl{\to} & \mathit{var}_1 \terminal{ = } \mathit{lf}_1\terminal{;} \ldots \terminal{;} \mathit{var}_n \terminal{ = } \mathit{lf}_n\terminal{;} & \bl{n \ge 1}\\
    \text{\bl{$\lambda$-forms}} & \mathit{lf} & \bl{\to} & \mathit{vars}_f~\terminal{\textbackslash} \pi~\mathit{vars}_a \terminal{ -> } \mathit{expr} \\
    \text{\bl{Update flag}} & \pi & \bl{\to} & \terminal{u} & \text{\bl{Updatable}} \\
    &     & \bl{\mid}     & \terminal{n} & \text{\bl{Not updatable}} \\
    \text{\bl{Expression}} & \mathit{expr} & \bl{\to}  & \terminal{let}~\mathit{binds}~\terminal{in}~\mathit{expr} & \text{\bl{Local definition}} \\
    &               & \bl{\mid} & \terminal{letrec}~\mathit{binds}~\terminal{in}~\mathit{expr} & \text{\bl{Local recursion}} \\
    &               & \bl{\mid} & \terminal{case}~\mathit{expr}~\terminal{of}~\mathit{alts} & \text{\bl{Case expression}} \\
    &               & \bl{\mid} & \mathit{var}~\mathit{atoms} & \text{\bl{Application}} \\
    &               & \bl{\mid} & \mathit{constr}~\mathit{atoms} & \text{\bl{Saturated constructor}} \\
    &               & \bl{\mid} & \mathit{prim}~\mathit{atoms} & \text{\bl{Saturated built-in operator}} \\
    &               & \bl{\mid} & \mathit{literal}  \\
    \text{\bl{Alternatives}} & \mathit{alts} & \bl{\to}  & \mathit{aalt}_1 \terminal{;} \ldots \terminal{;} \mathit{aalt}_n \terminal{;} \mathit{default} & \bl{n \ge 0} \text{ \bl{(Algebraic)}} \\
    &               & \bl{\mid} & \mathit{palt}_1 \terminal{;} \ldots \terminal{;} \mathit{palt}_n \terminal{;} \mathit{default} & \bl{n \ge 0} \text{ \bl{(Primitive)}} \\
    \text{\bl{Algebraic alt}} & \mathit{aalt} & \bl{\to}  & \mathit{constr}~\mathit{vars}~\terminal{->}~\mathit{expr} \\
    \text{\bl{Primitive alt}} & \mathit{palt} & \bl{\to}  & \mathit{literal}~\terminal{->}~\mathit{expr} \\
    \text{\bl{Default alt}} & \mathit{default} & \bl{\to}  & \mathit{var}~\terminal{->}~\mathit{expr} \\
    &                  & \bl{\mid} & \terminal{default ->}~\mathit{expr}\\
    \text{\bl{Literals}} & \mathit{literal} & \bl{\to}  & \terminal{0\#}~\bl{\mid}~\terminal{1\#}~\bl{\mid}~\ldots & \text{\bl{Primitive integers}}\\
    \text{\bl{Primitive ops}} & \mathit{prim} & \bl{\to}  & \terminal{+\#}~\bl{\mid}~\terminal{-\#}~\bl{\mid}~\terminal{*\#}~\bl{\mid}~\terminal{/\#}  & \text{\bl{Primitive integer ops}}\\     
    \text{\bl{Variable lists}} & \mathit{vars} & \bl{\to} & \terminal{\{}~\mathit{var}_1~\terminal{,} \ldots \terminal{,}~\mathit{var}_n~\terminal{\}} & \bl{n \ge 0}\\  
    \text{\bl{Atom lists}} & \mathit{atoms} & \bl{\to} & \terminal{\{}~\mathit{atom}_1~\terminal{,} \ldots \terminal{,}~\mathit{atom}_n~\terminal{\}} & \bl{n \ge 0}\\ 
    \text{\bl{Atoms}} & \mathit{atom}  & \bl{\to} & \mathit{var} ~\bl{\mid}~ \mathit{literal}            \\                        
    \end{array}
    \end{displaymath}
    \caption{STG language}
    \label{fig:stglang}
\end{mdframed}
\end{figure}

\end{document}
