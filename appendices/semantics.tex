\subsection{Operational Semantics}
\label{app:semantics}

Configurations of the Spineless Tagless G-Machine are six-tuples
\begin{displaymath}
\begin{array}{lcl}
\mathit{Config} & = & \langle \mathit{code}, \mathit{as}, \mathit{rs}, \mathit{us}, \mathit{h}, \sigma \rangle 
\end{array}
\end{displaymath}
where
\begin{itemize}
\item $\mathit{code}$ is the current ``instruction'' to be evaluated (see below)
\item $\mathit{as}$ is the argument stack, which contains values (see below)
\item $\mathit{rs}$ is the return stack, which contains continuations 
\item $\mathit{us}$ is the update stack, which contains update frames 
\item $h$ is the heap, which contains only closures 
\item $\sigma$ is the global environment, which maps the names of global definitions to memory addresses 
\end{itemize}

\subsubsection{Addresses}

Memory addresses $\mathit{a} \in \mathbb{N}$. Each location in memory can be occupied by exactly one closure.

\subsubsection{Global Environment}

The global environment $\sigma$ is a mapping of labels to memory addresses. In general, a program in the STG language is a collection of global definitions:
\begin{displaymath}
    \begin{array}{lcl}
    g_0 & = & \mathit{vs}_0~\terminal{\textbackslash}\pi~\mathit{xs}_0~\terminal{->}~\mathit{e}_0 \\
    \ldots \\
    g_n & = & \mathit{vs}_n~\terminal{\textbackslash}\pi~\mathit{xs}_n~\terminal{->}~\mathit{e}_n \\
    \end{array}
\end{displaymath} 
The corresponding global environment would be:
\begin{displaymath}
\begin{array}{lcl}
\sigma & = & \set{g_0 \mapsto a_0, \ldots, g_n \mapsto a_n}
\end{array}
\end{displaymath}
where $a_0 \ldots a_n$ are distinct memory addresses.

\subsubsection{Values}

\emph{Note: these aren't values in the operational semantics sense of the word, but rather different interpretations of integers in the operational semantics.}

We distinguish between address values (pointers) and primitive integer values:
\begin{center}
\begin{tabular}{p{1cm}p{6cm}}
$\mathit{Addr}~a$ & A memory address (pointer) \\
$\mathit{Int}~n$  & A primitive integer value
\end{tabular}
\end{center}

\subsubsection{Local Environments}

A local environment $\rho$ is a mapping of variable names to values. Such environments are created as the result of local bindings (\emph{e.g.} \texttt{let} and \texttt{letrec} bindings).

\subsubsection{Closures}

A closure is a structure on the heap, consisting of an arbitrary complex $\lambda$-form and a collection of free variables. There is one closure for every global definition and additional closures will be allocated as a program is evaluated.

\emph{Note: for the purpose of the interpreter, we never bother to remove closures from the heap and assume that there is an infinite amount of memory available.}

\subsubsection{Heap}

The heap $h$ is a mapping from memory address to closures. In general, a program in the STG language is a collection of global definitions:
\begin{displaymath}
    \begin{array}{lcl}
    g_0 & = & \mathit{vs}_0~\terminal{\textbackslash}\pi~\mathit{xs}_0~\terminal{->}~\mathit{e}_0 \\
    \ldots \\
    g_n & = & \mathit{vs}_n~\terminal{\textbackslash}\pi~\mathit{xs}_n~\terminal{->}~\mathit{e}_n \\
    \end{array}
\end{displaymath} 
The corresponding heap is:
\begin{displaymath}
\begin{array}{lcl}
h & = & \begin{array}[t]{llclll}
\{ & a_0 & \mapsto & (\mathit{vs}_0~\terminal{\textbackslash}\pi~\mathit{xs}_0~\terminal{->}~\mathit{e}_0) & \set{} & \\
   &     & \ldots  &  \\
   & a_n & \mapsto & (\mathit{vs}_n~\terminal{\textbackslash}\pi~\mathit{xs}_n~\terminal{->}~\mathit{e}_n) & \set{} & \}
\end{array}
\end{array}
\end{displaymath}
\emph{Note that global definitions shouldn't have free variables, so the collections of free variables for their closures on the heap are empty, indicated by the $\set{}$.}

\subsubsection{Code}

The code component of a configuration is one of the following states:
\begin{center}
\begin{tabular}{p{2.5cm}p{8cm}}
$\mathit{Eval}~e~\rho$             & Evaluate the expression $e$ in the local environment $\rho$ and apply its value to the arguments on the argument stack. The expression $e$ can be an arbitrarily complex expression in the STG language. \\
$\mathit{Enter}~a$                 & Apply the closure at address $a$ to the arguments on the argument stack. \\
$\mathit{ReturnCon}~c~\mathit{vs}$ & Return the constructor $c$ applied to the values $\mathit{vs}$ to the continuation on the return stack. \\
$\mathit{ReturnInt}~n$             & Return the primitive integer $n$ to the continuation on the return stack. 
\end{tabular}
\end{center}

\subsubsection{The Stacks}

There are three stacks in the STG machine.

\begin{itemize}
\item The argument stack 
\item The return stack
\item Whenever we encounter a lambda form whose update flag is set to \texttt{u}, \emph{i.e.} it can be updated, an \emph{update frame} is pushed onto the update stack. 
\end{itemize}

\subsubsection{Retriving the value of an atom}

Before we start defining the transition rules for the Spineless Tagless G-Machine, let us first define a little helper function which allows us to retrieve the values associated with atoms (see \autoref{fig:stglang}). If the atom is a primitive integer, then the corresponding STG value is also just a primitive integer. If the atom is a variable and it is an element of the domain of the local environment $\rho$, then we return the value associated with it. Otherwise, we try to look up its value in the global environment $\sigma$:
\begin{displaymath}
\begin{array}{lcll}
\mathit{val}~\rho~\sigma~\texttt{n} & = & \mathit{Int}~\texttt{n} & \\
\mathit{val}~\rho~\sigma~\texttt{x} & = & \rho~\texttt{x}   & \text{if}~\texttt{x} \in \mathit{dom}(\rho) \\
                                    &   & \sigma~\texttt{x} & \text{otherwise}
\end{array}
\end{displaymath}

\subsubsection{Application}

The first rule in the operational semantics concerns the application of some variable to some list of atoms. If we are in an $\mathit{Eval}$ state and the expression is of form $\mathit{f}~\mathit{xs}$ where $f$ is a variable, $\mathit{xs}$ is a list of atoms, and the value of $f$ is a memory address (see Rule \ref{eqn:varint} for the case where the value is an integer literal), then we push the values of the atoms in $\mathit{xs}$ onto the argument stack and move into an $\mathit{Enter}$ state for the closure represented by $f$:
\begin{mdframed}
\begin{equation}
\begin{array}{cl}
 & \langle \mathit{Eval}~(f~\mathit{xs})~\rho, \mathit{as}, \mathit{rs}, \mathit{us}, h, \sigma \rangle \\
 & \mathbf{where}~\mathit{val}~\rho~\sigma~f = \mathit{Addr}~a\\[0.25cm]
\Longrightarrow & \langle \mathit{Enter}~a, \overline{(\mathit{val}~\rho~\sigma~\mathit{xs})} \append \mathit{as}, \mathit{rs}, \mathit{us}, h, \sigma \rangle
\end{array}
\end{equation}
\end{mdframed}
The $\overline{\mathit{val}~\rho~\sigma~\mathit{xs}}$ notation is used to mean ``apply $\mathit{val}~\rho~\sigma$ to every element in the list $\mathit{xs}$''. The $\append$ operator is used to concatenate two lists.

The second rule concerns entering a closure. For now, we will only consider \emph{non-updatable} closures. The rule for updatable closures is given in Appendix \ref{app:updates}.
\begin{mdframed}
\begin{equation}
\begin{array}{cl}
 & \langle \mathit{Enter}~a, \mathit{as}, \mathit{rs}, \mathit{us}, h[a \mapsto \mathit{vs}~\terminal{\textbackslash}\pi~\mathit{xs}~\terminal{->}~\mathit{e}], \sigma \rangle \\
 & \mathbf{where}~\mathit{length}~\mathit{as} \geq \mathit{length}~\mathit{xs}\\[0.25cm]
\Longrightarrow & \langle \mathit{Eval}~e~\rho, \mathit{as}', \mathit{rs}, \mathit{us}, h, \sigma \rangle \\
 & \mathbf{where}~\begin{array}[t]{lcl}
  \mathit{ws}_a \append \mathit{as'} & = & \mathit{as} \qquad \textbf{such that}~\mathit{length}~\mathit{ws}_a = \mathit{length}~\mathit{xs} \\
  \rho & = & \hslist{\overline{\mathit{vs} \mapsto \mathit{ws}_f}, \overline{\mathit{xs} \mapsto \mathit{ws}_a}}
  \end{array}
\end{array}
\end{equation}
\end{mdframed}
When a non-updatable closure is entered, the local environment $\rho$ is constructed by binding the $\lambda$-form's free variables, $\mathit{vs}$, to the values, $\mathit{ws}_f$, found in the closure. The arguments, $\mathit{xs}$, are bound to the values found on top of the argument stack, $\mathit{ws}_a$. The resulting configuration indicates that the body of the closure should be evaluated next.

\subsubsection{\texttt{let(rec)} expressions}

A \texttt{let} expression constructs closures for all of its bindings on the heap:
\begin{mdframed}
\begin{equation}
\begin{array}{cl}
 & \langle \mathit{Eval}~\left( \begin{array}{llcl}
 \mathbf{let} & x_1 & = & \mathit{vs}_1~\terminal{\textbackslash}\pi_1~\mathit{xs}_1~\terminal{->}~\mathit{e}_1 \\
              & \multicolumn{3}{l}{\ldots} \\
              & x_n & = & \mathit{vs}_n~\terminal{\textbackslash}\pi_n~\mathit{xs}_n~\terminal{->}~\mathit{e}_n \\
 \multicolumn{4}{l}{\mathbf{in}~e}
 \end{array} \right)~\rho,\mathit{as},\mathit{rs},\mathit{us},h,\sigma \rangle \\[1.5cm]
\Longrightarrow & \langle \mathit{Eval}~e~\rho', \mathit{as},\mathit{rs},\mathit{us},h',\sigma \rangle \\
 & \mathbf{where}~\begin{array}[t]{lcl}
 \rho' & = & \rho[x_1 \mapsto \mathit{Addr}~a_1, \ldots, x_n \mapsto \mathit{Addr}~a_n] \\
 h' & = & h\left[\begin{array}{lcl}
 a_1 & \mapsto & (\mathit{vs}_1~\terminal{\textbackslash}\pi_1~\mathit{xs}_1~\terminal{->}~\mathit{e}_1)~(\rho_{\mathit{rhs}~\mathit{vs}_1}) \\
 \multicolumn{3}{l}{\ldots} \\
 a_n & \mapsto & (\mathit{vs}_n~\terminal{\textbackslash}\pi_n~\mathit{xs}_n~\terminal{->}~\mathit{e}_n)~(\rho_{\mathit{rhs}~\mathit{vs}_n})
 \end{array}\right] \\
 \rho_{\mathit{rhs}} & = & \rho
 \end{array}
\end{array}
\label{eqn:let}
\end{equation}
\end{mdframed}
First, we construct a new local environment, $\rho'$, by extending $\rho$ with mappings for all the binding names, $x_1 \ldots x_n$, to fresh addresses, $a_1 \ldots a_n$ -- \emph{i.e.} ones that have not been used for anything else yet. Then we construct a new heap, $h'$, by extending the old heap, $h$, with closures for all bindings. The free variables of the bindings are resolved using $\rho_{\mathit{rhs}}$, which is the same as the old local environment, $\rho$ -- \emph{i.e.} it does not yet contain mappings for the bindings in this \texttt{let} expression.

The rule for \texttt{letrec} is identical to Rule \ref{eqn:let}, except that $\rho_{\mathit{rhs}} = \rho'$ so that all bindings in the \texttt{letrec}-expression are in each others' scope:
\begin{mdframed}
\begin{equation}
\begin{array}{cl}
 & \langle \mathit{Eval}~\left( \begin{array}{llcl}
 \mathbf{letrec} & x_1 & = & \mathit{vs}_1~\terminal{\textbackslash}\pi_1~\mathit{xs}_1~\terminal{->}~\mathit{e}_1 \\
              & \multicolumn{3}{l}{\ldots} \\
              & x_n & = & \mathit{vs}_n~\terminal{\textbackslash}\pi_n~\mathit{xs}_n~\terminal{->}~\mathit{e}_n \\
 \multicolumn{4}{l}{\mathbf{in}~e}
 \end{array} \right)~\rho,\mathit{as},\mathit{rs},\mathit{us},h,\sigma \rangle \\[0.25cm]
\Longrightarrow & \langle \mathit{Eval}~e~\rho', \mathit{as},\mathit{rs},\mathit{us},h',\sigma \rangle \\
 & \mathbf{where}~\begin{array}[t]{lcl}
 \rho' & = & \rho[x_1 \mapsto \mathit{Addr}~a_1, \ldots, x_n \mapsto \mathit{Addr}~a_n] \\
 h' & = & h\left[\begin{array}{lcl}
 a_1 & \mapsto & (\mathit{vs}_1~\terminal{\textbackslash}\pi_1~\mathit{xs}_1~\terminal{->}~\mathit{e}_1)~(\rho_{\mathit{rhs}~\mathit{vs}_1}) \\
 \multicolumn{3}{l}{\ldots} \\
 a_n & \mapsto & (\mathit{vs}_n~\terminal{\textbackslash}\pi_n~\mathit{xs}_n~\terminal{->}~\mathit{e}_n)~(\rho_{\mathit{rhs}~\mathit{vs}_n})
 \end{array}\right] \\
 \rho_{\mathit{rhs}} & = & \rho'
 \end{array}
\end{array}
\label{eqn:let}
\end{equation}
\end{mdframed}

\subsubsection{Case expressions and data constructors}

The return stack is used for the first time when we come to \texttt{case} expressions. Given the expression
\begin{displaymath}
\texttt{case}~e~\texttt{of}~\mathit{alts}
\end{displaymath}
the operational interpretation is ``push a continuation onto the return stack, and evaluate $e$''. When the evaluation of $e$ is complete, execution will resume at the continuation, which then decides which alternative to execute. The rule for \texttt{case} follows fairly directly:
\begin{mdframed}
\begin{equation}
\begin{array}{cl}
 & \langle \mathit{Eval}~(\texttt{case}~e~\texttt{of}~\mathit{alts})~\rho, \mathit{as}, \mathit{rs}, \mathit{us}, h, \sigma \rangle \\[0.25cm]
\Longrightarrow & \langle \mathit{Eval}~e~\rho, \mathit{as}, (\mathit{alts}, \rho) : \mathit{rs}, \mathit{us}, h, \sigma \rangle 
\end{array}
\end{equation}
\end{mdframed}
The continuation is a pair $(\mathit{alts}, \rho)$; the alternatives $\mathit{alts}$ say what to do when evaluation of $e$ completes, while the local environment $\rho$ provides the context in which to evaluate the chosen alternative. 

The other side of the coin is the rules for constructors and literals. Presumably, $e$ eventually evaluates to either a constructor or a literal, at which point the continuation must be popped from the return stack and executed. The rules for constructors and literals each use an intermediate state, $\mathit{ReturnCon}$ and $\mathit{ReturnInt}$ respectively, just as the rule for function application uses $\mathit{Enter}$. Primitive values are dealt with in the next section, while the rules for constructors are given next.

Evaluating a constructor application simply moves into the $\mathit{ReturnCon}$ state:
\begin{mdframed}
\begin{equation}
\begin{array}{cl}
 & \langle \mathit{Eval}~(c~\mathit{xs})~\rho, \mathit{as}, \mathit{rs}, \mathit{us}, h, \sigma \rangle \\[0.25cm]
\Longrightarrow & \langle \mathit{ReturnCon}~c~(\overline{\mathit{val}~\rho~\sigma~\mathit{xs}}), \mathit{as}, \mathit{rs}, \mathit{us}, h, \sigma \rangle 
\end{array}
\end{equation}
\end{mdframed}
The rules for $\mathit{ReturnCon}$ return to the appropriate continuation, taken from the return stack:
\begin{mdframed}
\begin{equation}
\begin{array}{cl}
 & \langle \mathit{ReturnCon}~c~\mathit{ws}, \mathit{as},  (\ldots \texttt{;} c~\mathit{vs}~\texttt{->}~e \texttt{;} \ldots, \rho) : \mathit{rs}, \mathit{us}, h, \sigma \rangle \\[0.25cm]
\Longrightarrow & \langle \mathit{Eval}~e~\rho[\overline{\mathit{vs} \mapsto \mathit{ws}}], \mathit{as}, \mathit{rs}, \mathit{us}, h, \sigma \rangle
\end{array}
\label{eq:ctrmatch}
\end{equation}
\end{mdframed}
Provided that the continuation on the return stack contains a pattern $c~\mathit{vs}$ whose constructor $c$ is the same as that being evaluated, we just evaluate the right-hand side of that alternative, $e$, in the saved local environment, $\rho$, augmented with bindings for the pattern variables, $\mathit{vs}$, to the values of the actual arguments to $c$, represented by $\mathit{ws}$.

If there is no such alternative in the continuation on top of the return stack, then the default alternative is taken. If the default alternative does not bind a variable, then the rule is easy:
\begin{mdframed}
\begin{equation}
\begin{array}{cl}
 & \langle \mathit{ReturnCon}~c~\mathit{ws}, \mathit{as}, \left(\begin{array}{lcl}
 c_1~\mathit{vs}_1 & \texttt{->} & e_1 \texttt{;} \\
 \multicolumn{3}{l}{\ldots \texttt{;}} \\
 c_n~\mathit{vs}_n & \texttt{->} & e_n \texttt{;} \\
 \texttt{default} & \texttt{->} & e_d
 \end{array}, \rho\right) : \mathit{rs}, \mathit{us}, h, \sigma \rangle \\
 & \mathbf{where}~\forall i \in \mathbb{N}~\text{such that}~1 \leq i \leq n, c \neq c_i \\[0.25cm]
\Longrightarrow & \langle \mathit{Eval}~e_d~\rho, \mathit{as}, \mathit{rs}, \mathit{us}, h, \sigma \rangle
\end{array}
\end{equation}
\end{mdframed}
We simply move to a state where the expression on the right-hand side of the default alternative, $e_d$, is the next expression to evaluate.

However, if the default alternative binds a variable $v$, we need to allocate a constructor closure on the heap. We do this by extending the local environment with a mapping for $v$ to the address of a new closure, containing a $\lambda$-form whose free variables are the constructor's arguments and whose body is the constructor $c$ applied to those variables. That closure is then added to the old heap, $h$, to form a new heap, $h'$: 
\begin{mdframed}
\begin{equation}
\begin{array}{cl}
 & \langle \mathit{ReturnCon}~c~\mathit{ws}, \mathit{as}, \left(\begin{array}{lcl}
 c_1~\mathit{vs}_1 & \texttt{->} & e_1 \texttt{;} \\
 \multicolumn{3}{l}{\ldots \texttt{;}} \\
 c_n~\mathit{vs}_n & \texttt{->} & e_n \texttt{;} \\
 v & \texttt{->} & e_d
 \end{array}, \rho\right) : \mathit{rs}, \mathit{us}, h, \sigma \rangle \\
 & \mathbf{where}~\forall i \in \mathbb{N}~\text{such that}~1 \leq i \leq n, c \neq c_i \\[0.25cm]
\Longrightarrow & \langle \mathit{Eval}~e_d~\rho[v \mapsto a], \mathit{as}, \mathit{rs}, \mathit{us}, h', \sigma \rangle\\
 & \mathbf{where}~\begin{array}[t]{l}
 h' = h[a \mapsto (\mathit{vs}~\terminal{\textbackslash}\texttt{n}~\texttt{\set{}}~\terminal{->}~\mathit{c}~\mathit{vs})~\mathit{ws}] \\
 \mathit{vs}~\text{is a sequence of arbitrary, distinct variables} \\
 \mathit{length}~\mathit{vs} = \mathit{length}~\mathit{ws}
 \end{array}
\end{array}
\end{equation}
\end{mdframed}

\subsubsection{Built-in operations}

In this section we give the extra rules which handle primitive values. The rule for evaluating a primitive literal, $k$, enters the $\mathit{ReturnInt}$ state:
\begin{mdframed}
\begin{equation}
\begin{array}{cl}
 & \langle \mathit{Eval}~k~\rho, \mathit{as}, \mathit{rs}, \mathit{us}, h, \sigma \rangle \\[0.25cm]
\Longrightarrow & \langle \mathit{ReturnInt}~k, \mathit{as}, \mathit{rs}, \mathit{us}, h, \sigma \rangle 
\end{array}
\end{equation}
\end{mdframed}
A similar rule deals with the case where a variable bound to a primitive value is entered:
\begin{mdframed}
\begin{equation}
\begin{array}{cl}
 & \langle \mathit{Eval}~(f~\texttt{\set{}})~\rho[f \mapsto \mathit{Int}~k], \mathit{as}, \mathit{rs}, \mathit{us}, h, \sigma \rangle \\[0.25cm]
\Longrightarrow & \langle \mathit{ReturnInt}~k, \mathit{as}, \mathit{rs}, \mathit{us}, h, \sigma \rangle 
\end{array}
\label{eqn:varint}
\end{equation}
\end{mdframed}
Next come the rules for the $\mathit{ReturnInt}$ state, which look for a continuation on the return stack. First, the case where there is an alternative which matches the literal (this corresponds to Rule \ref{eq:ctrmatch} for constructors):
\begin{mdframed}
\begin{equation}
\begin{array}{cl}
 & \langle \mathit{ReturnInt}~k, \mathit{as}, (\ldots \texttt{;} k~\texttt{->}~e \texttt{;} \ldots, \rho) : \mathit{rs}, \mathit{us}, h, \sigma \rangle \\[0.25cm]
\Longrightarrow & \langle \mathit{Eval}~e~\rho, \mathit{as}, \mathit{rs}, \mathit{us}, h, \sigma \rangle 
\end{array}
\end{equation}
\end{mdframed}
Next, the cases where the default alternative is taken:
\begin{mdframed}
\begin{equation}
\begin{array}{cl}
 & \langle \mathit{ReturnInt}~k, \mathit{as}, \left(\begin{array}{lcl}
 k_1 & \texttt{->} & e_1 \texttt{;} \\
 \multicolumn{3}{l}{\ldots \texttt{;}} \\
 k_n & \texttt{->} & e_n \texttt{;} \\
 x & \texttt{->} & e_d
 \end{array}, \rho\right) : \mathit{rs}, \mathit{us}, h, \sigma \rangle \\
 & \mathbf{where}~\forall i \in \mathbb{N}~\text{such that}~1 \leq i \leq n, k \neq k_i \\[0.25cm]
\Longrightarrow & \langle \mathit{Eval}~e_d~\rho[x \mapsto \mathit{Int}~k], \mathit{as}, \mathit{rs}, \mathit{us}, h, \sigma \rangle
\end{array}
\end{equation}
\end{mdframed}
\begin{mdframed}
\begin{equation}
\begin{array}{cl}
 & \langle \mathit{ReturnInt}~k, \mathit{as}, \left(\begin{array}{lcl}
 k_1 & \texttt{->} & e_1 \texttt{;} \\
 \multicolumn{3}{l}{\ldots \texttt{;}} \\
 k_n & \texttt{->} & e_n \texttt{;} \\
 \texttt{default} & \texttt{->} & e_d
 \end{array}, \rho\right) : \mathit{rs}, \mathit{us}, h, \sigma \rangle \\
 & \mathbf{where}~\forall i \in \mathbb{N}~\text{such that}~1 \leq i \leq n, k \neq k_i \\[0.25cm]
\Longrightarrow & \langle \mathit{Eval}~e_d~\rho, \mathit{as}, \mathit{rs}, \mathit{us}, h, \sigma \rangle
\end{array}
\end{equation}
\end{mdframed}
Finally, we need a family of rules for built-in arithmetic operations which, for each binary built-in operation $\oplus$ have the form:
\begin{mdframed}
\begin{equation}
\begin{array}{cl}
 & \langle \mathit{Eval}~(\oplus~\texttt{\set{$x_1\texttt{,} x_2$}})~\rho[x_1 \mapsto \mathit{Int}~i_1, x_2 \mapsto \mathit{Int}~i_2], \mathit{as}, \mathit{rs}, \mathit{us}, h, \sigma \rangle \\[0.25cm]
\Longrightarrow & \langle \mathit{ReturnInt}~(i_1 \oplus i_2)~\rho, \mathit{as}, \mathit{rs}, \mathit{us}, h, \sigma \rangle 
\end{array}
\end{equation}
\end{mdframed}
The $\oplus$ symbol is used to mean one of \texttt{+\#}, \texttt{-\#}, \texttt{*\#}, or \texttt{/\#} as well as the corresponding arithmetic operators, respectively.

\subsubsection{Updates}
\label{app:updates}

Updates happen in two stages:
\begin{enumerate}
\item When an updatable closure is entered, it pushes an \emph{update frame} onto the update stack, and makes the argument and return stacks empty. An update frame is a triple $(\mathit{as}_u, \mathit{rs}_u, a_u)$, consisting of:
\begin{itemize}
\item $\mathit{as}_u$, the previous argument stack;
\item $\mathit{rs}_u$, the previous return stack;
\item $a_u$, the pointer to the closure being entered, and which should later be updated.
\end{itemize}
\item When evaluation of the closure is complete, an update is triggered. This can happen in one of two ways:
\begin{itemize}
\item If the value of the closure is a data constructor or literal, an attempt will be made to pop a continuation from the return stack, which will fail because the return stack is empty. This failure triggers an update.
\item If the value of the closure is a function, the function will attempt to bind arguments which are not present on the argument stack, because they were moved into the update frame. This failure to find enough arguments triggers an update.
\end{itemize}
\end{enumerate}
If we attempt to enter a closure whose update flag is set to \texttt{u} to indicate that it can be updated, we push an update frame onto the update stack:
\begin{mdframed}
\begin{equation}
\begin{array}{cl}
 & \langle \mathit{Enter}~a, \mathit{as}, \mathit{rs}, \mathit{us}, h[a \mapsto (\mathit{vs}~\terminal{\textbackslash}\terminal{u}~\set{}~\terminal{->}~\mathit{e})~\mathit{ws}], \sigma \rangle \\[0.25cm]
\Longrightarrow & \langle \mathit{Eval}~e~\rho, \set{}, \set{}, (\mathit{as},\mathit{rs},a) : \mathit{us}, h, \sigma \rangle \\
 & \mathbf{where}~\begin{array}[t]{lcl}
 \rho & = & \hslist{\mathit{vs}_0 \mapsto \mathit{ws}_0, \ldots, \mathit{vs}_n \mapsto \mathit{ws}_n}
 \end{array}
\end{array}
\end{equation}
\end{mdframed}
Next, we need rules for constructors which see an empty return stack. When this happens, they update the closure pointed to by the update frame, restore the argument and return stacks from the update frame, and try again. It may be that the restored return stack contains the continuation, but it too may be empty, in which case a second update is performed, and so on until the continuation is exposed:
\begin{mdframed}
\begin{equation}
\begin{array}{cl}
 & \langle \mathit{ReturnCon}~c~\mathit{ws}, \set{}, \set{}, (\mathit{as}_u, \mathit{rs}_u, a_u) : \mathit{us}, h, \sigma \rangle \\[0.25cm]
\Longrightarrow & \langle \mathit{ReturnCon}~c~\mathit{ws}, \mathit{as}_u, \mathit{rs}_u, \mathit{us}, h_u, \sigma \rangle \\
 & \mathbf{where}~\begin{array}[t]{lcl}
 \multicolumn{3}{l}{\mathit{vs}~\text{is a sequence of arbitrary, distinct variables}} \\
 \multicolumn{3}{l}{\mathit{length}~\mathit{vs} = \mathit{length}~\mathit{ws}} \\
 h_u & = & h[a_u \mapsto (\mathit{vs}~\terminal{\textbackslash}\texttt{n}~\set{}~\terminal{->}~\mathit{c}~\mathit{vs})~\mathit{ws}]
 \end{array}
\end{array}
\end{equation}
\end{mdframed}
The closured to be updated at address $a_u$ is just updated with a standard constructor closure. Only a rule for $\mathit{ReturnCon}$ needs to be given. It is not possible for the $\mathit{ReturnInt}$ state to see an empty return stack, because that would imply that a closure should be updated with a primitive value, but no closure has a primitive type.

Finally, we need a rule to handle the case where there are not enough arguments on the stack to be bound by a $\lambda$-abstraction ($\mathit{length}~\mathit{as} < \mathit{length}~\mathit{xs}$), which triggers an update:
%Note that the pre-condition for this rule is the same as for Rule \ref{}, but with $\mathit{length}~\mathit{as} < \mathit{length}~\mathit{xs}$: 
\begin{mdframed}
\begin{equation}
\begin{array}{cl}
 & \langle \mathit{Enter}~a, \mathit{as}, \set{}, (\mathit{as}_u, \mathit{rs}_u, a_u) : \mathit{us}, h[a \mapsto \mathit{vs}~\terminal{\textbackslash}\terminal{n}~\mathit{xs}~\terminal{->}~\mathit{e}], \sigma \rangle \\
 & \mathbf{where}~\mathit{length}~\mathit{as} < \mathit{length}~\mathit{xs}\\[0.25cm]
\Longrightarrow & \langle \mathit{Enter}~a, \mathit{as} \append \mathit{as}_u, \mathit{rs}_u, \mathit{us}, h_u, \sigma \rangle \\
 & \mathbf{where}~\begin{array}[t]{lcl}
 \mathit{xs}_1 \append \mathit{xs}_2 & = & xs \qquad \textbf{such that}~\mathit{length}~\mathit{xs}_1 = \mathit{length}~\mathit{as} \\
 h_u & = & h[(a_u \mapsto (\mathit{vs} \append \mathit{xs}_1)~\terminal{\textbackslash}\terminal{n}~\mathit{xs}_2~\terminal{->}~e)~(\mathit{ws}_f \append \mathit{as})]
 \end{array}
\end{array}
\end{equation}
\end{mdframed}
